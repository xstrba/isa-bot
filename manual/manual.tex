\documentclass[a4paper,11pt]{article}

\usepackage[left=2cm,text={17cm, 24cm},top=3cm]{geometry}
\usepackage[czech]{babel}
\usepackage[utf8]{inputenc}
\usepackage{times}
\usepackage[colorlinks]{hyperref}
\usepackage{url}

\begin{document}
\begin{titlepage}
	\begin{center}
		\textsc{
		{\Huge Vysoké učení technické v Brně}\\{\huge Fakulta informačních technologií}
		}
		\vspace{\stretch{0.382}}
		{\LARGE \\ Síťové aplikace a~správa sítí -- Projekt \\[6pt] {\Huge Discord bot}}
		\vspace{\stretch{0.618}}
	\end{center}

	{\Large
	\today
	\hfill
	Boris Štrbák (xstrba05)
	}
\end{titlepage}

\tableofcontents
\newpage

\section{Úvod}
\subsection{Zadanie}
Vytvořte program isabot, který bude působit jako bot na komunikační službě Discord. Bot se připojí na Discord server na kanál "\#isa-bot" a bude reagovat na všechny zprávy zaslané ostatními uživateli. Bot bude fungovat jako echo pro všechny zprávy, které zachytí. V případě, že bot na daném kanále zachytí jakoukoli zprávu jiného uživatele (tedy jinou než svou vlastní) a zároveň, která není jiného bota (uživatelské jméno neobsahuje podřetězec "bot"), odešle tuto zprávu zpátky na kanál a to ve formátu "echo: $\langle$username$\rangle$ - $\langle$message$\rangle$" (kde $\langle$username$\rangle$ představuje uživatelské jméno uživatele, který odeslal původní zprávu).

\subsection{Spuštění programu}
Použití: isabot [-h$|$-{}-help] [-v$|$-{}-verbose] -t\ $\langle$bot\_access\_token$\rangle$

Pořadí parametrů je libovolné. Popis parametrů:

\begin{itemize}
	\item Spuštění programu bez parametrů zobrazí nápovědu.
	\item {-h$|$-{}-help : Vypíše nápovědu na standardní výstup.}
	\item {-v$|$-{}-verbose : Bude zobrazovat zprávy, na které bot reaguje na standardní výstup ve formátu "$\langle$channel$\rangle$~-~$\langle$username$\rangle$~:~$\langle$message$\rangle$".}
	\item -t $\langle$bot\_access\_token$\rangle$ : Zde je nutno zadat autentizační token pro přístup bota na Discord.
\end{itemize}

\subsection{Implementačné nástroje}
Bot je implementovaný v jazyku \texttt{C++} za pomoci knižnice \texttt{openssl/ssl.h}.

\section{Úvod do problematiky}

\subsection{Discord bot}
Discord bot je typ uživatele v aplikaci discord, který není reálnym uživatelem ale je mu přiřazen token,
pomocí kterého se dokáže autorizovat v api poskytaveném discordem a umožňuje tak developerům různých aplikací například vytvářet nebo číst správy v kanálech discordu.
Pro použití programu je potřeba takového bota vytvořit \cite{dbot}. Při přidávaní bota na server je třeba nastavit práva na: View Channels, Embed Links, Read Message History, Create message.

\subsection{Komunikace pomocí soketu}
Pro zasílaní dotazů a získavaní odpovědí pomocí api discord serveru je potřeba navázat spojení pomocí soketu. Pro zasílaní jednoduchých HTTP dotazů by stačilo vytvořit soket pomocí knižnice \texttt{sys/socket.h}\cite{socketc} a připojit se na discord server na portu 80. Dicord server ale komunikuje pomocí protokolu HTTPS takže potřebujeme navázat
zabezpečené spojení pomocí SSL a na portu 443. Na to se dá využit knižnice \texttt{openssl/ssl.h}\cite{ssldocs} a příklad implementace HTTPS klienta se da najít například zde\cite{ssli}.

\subsection{Discord API}
Api discordu poskytuje celou řadu endpointů. Aplikace využíva jenom ty spomínané v nasledujících podkapitolách.
Pro používaní api je ale potřeba se autorizovat. To se děje pomocí tokenu bota, který je předán přes konzoli a ku každému požadavku na server je přidána hlavička \texttt{Authorization: Bot $\langle$bot\_access\_token$\rangle$}.
Discord poskytue dva druhy API. HTTP a Gateway. Aplikace používa HTTP API ale pro správne fungovaní je potřeba bota alespoň jednou připojit přes Gateway\cite{wsconnect}.

\subsubsection{Guildy}
\texttt{GET /api/users/@me/guilds HTTP/1.1}\\\\
Tento endpoint vrací pole všech guild, kterých je uživatel členem.

\subsubsection{Kanály}
\texttt{GET /guilds/\{guild.id\}/channels HTTP/1.1}\\\\
Tento endpoint vrací pole všech kanálu v rámci dané guildy. Aplikace se připojuje na kanál s názvem isa-bot.

\subsubsection{Správy}
\texttt{GET /channels/\{channel.id\}/messages?after=\{message.id\} HTTP/1.1}\\\\
Tento endpoint vrací pole všech správ v rámci daného kanálu. Správy jsou seřazeny od nejnovějši po nejstarší. Ku query je možné dopsat query parametr \texttt{after} s hodnout ID správy a api vrátí jenom správy vytvořené po této správě.
Každá správa ma ako parametr jméno uživatele, pomocí kterého se da skontrolovat či neobsahuje podřetězec "bot". Také obsahuje paramter "user", který obsahuje parametr "bot", který může být \texttt{true} nebo \texttt{false}.

\subsubsection{Odeslání správy}
\texttt{POST /channels/\{channel.id\}/messages HTTP/1.1}\\\\
Tento endpoint slouží pro vytvoření nové správy  rámci daného kanálu.

\section{Návrh aplikace}
Aplikace nejdříve spracuje argumenty předané přes konzoli. Pak se pokusí vypsat všechny guildy, ke kterým je bot přidán. Vybere první a z ní vypíše všechny kanály. Pokusí se najít textový kanál "isa-bot". Pokud ho najde tak spustí nekonečnou smyčku, ve které nejdříve získá nové správy a pak pro každou vytvoří novou v požadovaném formátu.

\section{Implementace}
\subsection{Kontrola argumentů}
Kontroloa argumentů probíha ve funkci \texttt{processArgs}, která projde všechny argumenty programu a v případě, že je nějaký špatný vráti false. V případě že najde argument "-t", bere dalši argument jako token ať je to cokoli.

\subsection{Vytvoření spojení přes soket}
Spojení přes soket se vytváři pomocí třidy \texttt{DiscordSocket}. Po vytvoření instance třídy se zavolá metoda \texttt{initialize}. Tá pomocí funkce \texttt{getaddrinfo} zistí IPv4 adresu discord serveru a pak vytvoří soket pomocí funkcí z knižnice \texttt{sys/socket.h} \cite{socketc} a ssl spojení pomocí funkcí z kninice \texttt{openssl/ssl.h} \cite{ssldocs}.

\subsection{Získaní ID kanálu}
Po vytvoření spojení se bot pokusí získat ID kanálu, na kterém bude zachytavát a posílat správy. Nejdřivé odešle dotaz na ednpoint pro získaní guildy a pak pro získaní kanálu. V poli kanálů s epokusí najít kanál "isa-bot" a zkontroluje zda je textový podle parametru "type". Tohle vše se děje ve funkci \texttt{setChannelId} třídy \texttt{DiscordBot}.
Této třídě je při vytvoření předán token bota, dřívě vytvořen soket a parametr "verbose", který je \texttt{true} nebo \texttt{false}. Objekt kanálu, který je v odpovědi serveru taktéž obsahuje ID poslední správy. To je uložené v property \texttt{lastMsgId} třídy DiscordBot.

\subsection{Spracování správ}
Po získaní ID kanálu se spustí nekonečná smyčka. Nejdříve se zavolá funkce \texttt{loadNewMessages} třidy DiscordBot, která načte nové správy a uloží ID posledné. Pak se zavolá funkce \texttt{reactToMessages} třídy DiscordBot, která spracuje všechny načtené správy a pro každou znovu uloži novú hodnotu \texttt{lastMsgId} pro případ, že by se nepodařilo je všechny zpracovat. Mezi načítaním nových správ je pauza 1 sekundu. Mezi zpracovaním jednotlivých správ a odesílaním požadavků na server je pauza 200 ms. Bot by tak neměl vyčerpat limity pro počet requestů za určitou dobu. V případě, že se to i tak stane tak server vrátí odpověd \texttt{429 (TOO MANY REQUESTS)}. Pak bot přečte z odpovědi kolik sekund má čekat a tolik i čeká před pokračovaním ve vykonávaní své funkce.

\subsection{Třída HttpResponse}
Táto třída slouži pro spracovaní odpovědi discord serveru. Spracuje první řádek obsahující protokol, kód a status.
Pak spracuje hlavičky a uloží ich do hashovací tabulky. Poté předá spracovaní těla odpovědi tříde \texttt{JsonParser}.

\subsection{Třída JsonParser}
Táto třída spracovává tělo odpovědi discord serveru. Jednotlivé entity ukládá pomocí jednoduchého automatu do třídy \texttt{JsonValue}.

\subsection{Třída JsonValue}
Táto třída slouží pro uložení JSON hodnot v těle odpovědi serveru. Ukláda jenom pole a objekty a všechny ostatné hodnoty ukládá jako řetězce protože to pro funkcionalitu bota bohatě stačí.

\section{Chybové hlášky}
\subsection{Error: Nepodařilo se připojít na server}
Chybová hláška v případě, že se nepodařilo spojit se serverem discord.com.

\subsection{Error: Nepodařilo se vytvořit spojení pomocí soketu}
Chybová hláška v případě, že se nepodařilo vytvořit soket.

\subsection{Error: Nepodařilo se vytvořit ssl spojení}
Chybová hláška v případě, že se nepodařilo vytvořit ssl spojení.

\subsection{Error: Nepodařilo se odeslat data}
Chybová hláška v případě, že se nepodařilo poslat data přes soket. Nespůsobí ukončení programu.

\subsection{Error: Chyba při spracovaní paketu}
Chybová hláška v případě, že se nepodařilo spracovat data paketu. Nespůsobí ukončení programu.

\subsection{Error: Nepodařilo se uzavřít spojení se serverem}
Chybová hláška v případě, že se nepodařilo ukončit spojení se serverem pomocí requestu s hlavičou \texttt{Connection: close}.

\subsection{Error: Chyba při čtení paketu}
Chybová hláška v případě, že se nepodařilo přečíst paket. Nespůsobí ukončení programu.

\subsection{Error: Chyba autorizace. Zkontrolujte token}
Chyba nastane pokud api vrátí odpověd 401. Teda nejspíš je špatný token.

\subsection{Error: Bot není připojenej na žádnej server}
Chyba nastane pokud api nevrátí žádnou guildu.

\subsection{Error: Nebyl nalezen kanál isa-bot}
Chyba nastane pokud api nevrátí textový kanlál isa-bot.

\subsection{Error: Došlo ku chybe pri spracovaní odpovědí discord serveru}
Chyba, která nastane pokud api nevrátí non-well formed data.

\subsection{Error: Došlo ku chybe při komunikaci s discord serverem}
Nastane při chybě na straně discord serveru. Api vráti 500 nebo 503...

\subsection{Error: Bot nemá dostatečné práva. Povolte práva View Channels, Send Messages, Read Message History a Embed Links z rozsahu Bot}
Chyba nastane pokud api vrátí 403, tedy bot nemá nastavená dostatečná práva.

\subsection{Error: Dotaz na neexistujíci url}
Interní chyba, která nastane při požadavku na špatnou url. Nemělo by se stát.

\subsection{Error: Dotaz špatnou metodou}
Interní chyba, která nastane při požadavku špatnou metodou. Nemělo by se stát.

\subsection{Error: Neznáma chyba při práci bota}
Interní chyba, která se nepodařila zachytit. Nemělo by se stát.

\section{Testování}
Testování probíhalo pomocí testovacího bota a kanálu na discord serveru a taktéž pomocí unit testů implementovaných v souboru \texttt{tests.cpp}. Unit testy byli napsané pro třidy JsonValue, JsonParser a HttpResponse.

\section{Použití programu}
\subsection{Překlad}
Překlad a kompilace programu proběhne zadaním příkazu \texttt{make}.

\subsection{Spuštění testů}
Spuštění unit testů proběhne zadaním příkazu \texttt{make test}.

\subsection{Spuštění programu}
Před spuštěním je potřeba nejdříve program přeložit. Pak se spustí příkazem \texttt{./isabot -v -t $\langle$bot\_access\_token$\rangle$}

\subsection{Zobrazení nápovědy}
Pro zobrazení nápovědy je třeba spustit program s přepínačem "-h" nebo "--help". Například \texttt{./isabot -h}

\subsection{Balík pro odevzdání}
Pomocí příkazu \texttt{make pack} se vytvoří balík pro odevzdání.

\subsection{Vyčištení adresáře}
Pomocí příkazu \texttt{make clean} se odstrání všechny vygenerované soubory z předešlých příkazů.

\newpage

\begin{thebibliography}{6}
	\bibitem{dbot} Rapptz. (2015 - 2020). Creating a Bot Account. Dostupné na: \\ \url{https://discordpy.readthedocs.io/en/latest/discord.html}
	\bibitem{dapi} Discord developer team. Official Discord API Documentation. Dostupné na: \\ \url{https://discord.com/developers/docs/intro}
	\bibitem{socketc} Silver, Moon. (17. máj 2020). Socket programming in C on Linux – The Ultimate Guide for Beginners. Dostupné na: \\ \url{https://www.binarytides.com/socket-programming-c-linux-tutorial/}
	\bibitem{ssldocs} OpenSSL Software Foundation. (1999-2018). OpenSSL official documentation. Dostupné na: \\ \url{https://www.openssl.org/docs/}
	\bibitem{ssli} Kalin, Martin. (19. jún 2019). Getting started with OpenSSL: Cryptography basics. Dostupné na: \\ \url{https://opensource.com/article/19/6/cryptography-basics-openssl-part-1}
	\bibitem{wsconnect} aequasi. Gateway connector. Dostupné na: \\ \url{https://github.com/restcord/restcord/blob/master/extra/gateway.html}
\end{thebibliography}

\end{document}